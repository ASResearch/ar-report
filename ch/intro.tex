% !TEX root = main.tex

\section{概要}

区块链技术带来的“去中心化”理念正在被用于越来越多的场景。作为区块链技术的起源,比特币已经证实了去中心化对于
数字资产的非凡的意义;更进一步的,以太坊证实了去中心化对于分布式应用的重要性;越来越多的区块链项目正在探索
去中心化这一理念在更多场景及应用下的价值。
不难发现,“去中心化”理念的背后,是区块链系统中的公开透明性与匿名性并存,这使得区块链系统在一定程度上为
链上的交易和数据提供了安全保障,使得任何组织和个人无法在不获得用户私钥情况下修改用户的数据,或判断数据
属于哪一个用户。

然而,区块链系统的公开透明性匿名性在一定程度上造成了价值衡量体系的缺失~\cite{meiklejohn2013fistful}。这反映在两方面,
首先,区块链系统中数据维度较弱,无法推断多个属于不同账户的数据和资产是否属于同一个用户,这导致了区块链系统中不能构建类似Cookie
的系统,也很难通过传统的数据分析技术从不同的角度分析用户特征;其次,区块链系统的公开透明性使得其面临着很强的操纵挑战,价值衡量体系很容易
受到各种针对性的操纵攻击,这不同于任何封闭的、独立的价值衡量体系。

我们认为有效的价值衡量体系是区块链系统之上的生态能够繁荣的基础,价值衡量体系的缺失或无效必然会限制整个区块链行业的发展。
首先,价值衡量体系为区块链系统、之上的应用、数据及账户的价值提供了
可以判断的量化标准,这是更大规模的、更高效的协作不可或缺的。其次,有效的价值衡量体系为更进一步的激励提供了重要的根据,
有效的价值衡量体系意味着有效的激励及健康的生态发展方向,反之,歪曲的价值衡量体系将会使整个区块链系统不可避免的走向灭亡。
最后,为进一步发掘区块链上数据及资产的价值,
许多应用均需要使用到相应的价值衡量标准,例如,区块链上的借贷和
征信类服务、数据搜索和个性化推荐、原生跨链交易和数据交换等,有效的价值衡量体系直接地促使了这些应用在区块链系统中的出现。
综上,这意味这一个价值衡量标准需要具备三个特点:
\begin{itemize}
\item{真实性} 一个好的价值衡量标准应该能够准确反映出区块链经济系统的特征,这样才能在相应的领域具有
足够的公信力;
\item{公平性} 价值衡量标准为相应的激励提供了依据,因此,这一
依据必须足够公平,才能防止作弊或操纵带来的“劣币驱逐良币”现象;
\item{多样性} 需要使用数据及数字资产价值的场景可能是多种多样的,其使用方式及对应的激励方式不尽相同,
因此相应的价值衡量标准既不能脱离应用场景,亦要满足前述的真实性及公平性。
\end{itemize}


我们使用星云指数(Nebulas Rank)衡量账户地址对于区块链这一经济系统的贡献度。
在经典的货币理论的基础上,我们认为区块链系统之上的加密数字货币应该具备基本的货币属性,并且,加密数字货币的价值源于其流通性。
因此,我们认为加密数字货币的交易记录是衡量加密数字货币这一经济体的有效数据来源。更进一步的,我们认为每个账户发
起的每一笔交易都在一定程度上增加了加密数字货币的流通性。微观地,每个账户的交易行为都最终反映在了区块链系统的价值中;
宏观地,我们将所有账户地址的星云指数的和定义为整个区块链系统的经济总量。

在星云指数的理论基础上,我们将星云指数分为核心星云指数(\nrcore)和计算基于核心星云指数的扩展星云指数(\nrext)两部分。其中,核心星云指数针对
区块链中不同账户对于区块链系统的贡献度给出了计算方法,此为基石;扩展星云指数则基于核心星云指数,针对区块链生态中各应用可能
需要的价值尺度给出了不同的计算方法。核心星云指数的计算基于两个参考因素,账户在一定时期内的资产中值及账户在一定时期内
的出入度衡量,特别的,我们设计了能够有效抵抗操纵的计算函数。更进一步的,为了验证星云指数的有效性,我们在基于以太坊的链上数据中计算了所有账户的星云指数之和,
并与Coinmarketcap中同期的以太坊市值进行了对比,我们的对比表明了二者具有很强的正相关性(0.85),即星云指数既能够在微观层面衡量每个账户
对经济系统的贡献,亦能在宏观层面反映整个经济系统总量的变化。最后,我们的分析表明了星云指数在抵抗操纵方面的性能。


进一步的,我们讨论了星云指数在实现方面的考量,例如是否将星云指数上链,星云指数的计算如何更新等。更进一步的,我们讨论了几种可能的扩展星云指数的计算方法,例如,如何根据核心
星云指数对智能合约进行排名;如何将星云指数拓展到多维并给予不同的权重,以适应不同的应用场景。最后,我们给出了下一步工作的方向。



\whitepaper{
特殊提示:
本文的描述与星云技术白皮书(2018年4月发布的1.02版本)~\cite{Nabulas}中对星云指数的描述有所出入,
这是因为经过一年的深入思考与验证,我们有信心和能力提出更为严谨的算法。
为了加以区分和比较,我们使用实线框高亮了技术白皮书中不够严谨的章节,并给出进一步的说明。
}
