\section{概要}
%作为加密数字货币的基石,区块链技术带给我们真正意义上的“去中心化的交易”


区块链提供了天生的匿名性,这种匿名性在一定程度上为区块链之上的交易和数据提供了安全保障,
然而也在一定程度上为进一步发掘区块链上数据及资产的价值带来了挑战。由于区块链的匿名性,
我们无从推断多个属于不同账户的数据和资产是否属于同一个用户,这种关联性的丧失,一定程度
上是区块链所追求的目标,同时,也造成了一些增值应用的缺失。例如,在区块链上的类似的借贷、
征信,区块链上的数据搜索、个性化推荐,更进一步的,区块链上原生的跨链交易、数据交换等,
这些应用均需要相应的价值衡量体系。

然而,一个更加本质的问题是:如何定义区块链上的数据及资产的价值,这种价值又如何量化。
我们认为,价值的量化是依赖于价值所使用的场景的。例如区块链上的去中心化应用推荐系统需要考量
相应的数据的价值,以便作出更好的推荐,作为对比,跨链的数据交换也需要衡量相应的数据的价值,
这两种场景下的对数据的价值的衡量,很难采用统一的标准。

建立通用的价值衡量标准是十分困难的,这种主要体现在以下几个方面:
\begin{itemize}
\item{多样性} 需要使用数据及数字资产价值的场景可能是多种多样的,其使用方式及对应的激励方式,
都是不尽相同的;
\item{真实性} 一个好的价值衡量标准,应该能够准确反映出至少某一方面的特征,这样才能在相应的方向具有
足够的公信力;
\item{公平性} 这是价值衡量标准的根本所在。可以想象,价值衡量标准为相应的激励提供了依据,因此,这一
依据必须足够公平,才能防止作弊或操纵带来的“劣币驱逐良币”。
\end{itemize}

我们认为,虽然一个通用的价值衡量标准很难建立,但是在区块链上量化数据或数字资产的方法是十分
关键的,因此,我们试图在星云链的场景中,给出一个建立价值尺度的方法,即星云指数,Nebulas Rank。
星云指数对于如何建立一个有效的价值衡量尺度提供了完整的方法论支持,从三个方面解决了建立通用价值衡量
标准的困难:
\begin{itemize}
\item{多样性} 星云指数包括核心星云指数,\nrcore,及扩展星云指数,\nrext。其中,核心星云指数针对
星云链中的不同账户对与星云链的贡献度给出了计算方法;扩展星云指数则针对星云链生态中的应用可能
需要的价值尺度给出了计算方法。需要注意的是,扩展星云指数的计算基于核心星云指数。
由此,星云指数通过不同的价值计算方法为不同的应用提供价值衡量方法,从而
满足了多样性需求;
\item{真实性} 我们提出的核心星云指数表明了账户地址对于星云链的贡献度,我们的计算表明,核心星云指数
与星云链的价值呈现出了很强的正相关性,也就是说,真实地反映了星云链的价值。因此,我们认为,核心星云
指数的真实性是有所保证的,相应的,其上的扩展星云指数的真实性也是同样的。
\item{公平性} {\color{blue} 为了抵抗操纵,我们引入了博弈论,所设计的星云指数“效用函数”在纳什均衡下,不存在“操纵”
的空间。}
\end{itemize}


需要注意的是,本文的描述与技术白皮书中对于星云指数的描述有所出入,这是因为我们对于星云指数进行了更加深入的思考及验证,
提出了更为严谨的算法。更具体的,我们对于技术白皮书中不够严谨的章节,使用了{\color{gray}特别的颜色},给予了进一步的说明。
