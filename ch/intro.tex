% !TEX root = main.tex

\section{概要}

区块链技术带来的“去中心化”理念正在被用于越来越多的场景。作为区块链技术的起源,比特币已经证实了去中心化对于
数字资产的非凡意义;更进一步的,以太坊证实了去中心化对于分布式应用的重要性;越来越多的区块链项目正在探索
去中心化这一理念在更多场景及应用下的价值。

不难发现,“去中心化”理念的背后,是区块链系统中的开放性(Openness)与匿名性并存。
%,这使得区块链系统在一定程度上为
%链上的交易和数据提供了安全保障,使得任何组织和个人无法在不获得用户私钥情况下修改用户的数据,并且很难判断数据
%属于哪一个用户。

然而,区块链系统的这种特性在一定程度上造成了价值衡量体系的缺失~\cite{meiklejohn2013fistful}。这反映在两方面。
首先,由于区块链系统的匿名性,很难推断多个属于不同账户的数据和资产是否属于同一个用户,这导致了区块链系统中不能构建类似HTTP Cookie~\cite{Cookie}
的机制,也很难通过传统的数据分析技术从不同的角度分析用户特征;另一方面,区块链系统的开放性又使得其面临着很强的操纵挑战,价值衡量体系很容易
受到各种针对性的操纵攻击,这不同于任何封闭的、独立的价值衡量体系。

我们认为有效的价值衡量体系是区块链生态能够繁荣发展的基础,价值衡量体系的缺失或无效必然会限制整个区块链行业的发展。

首先,随着协作规模不断变大,并且对效率的要求不断升级,我们需要一个价值衡量体系来为区块链系统以及区块链系统上的应用、数据和账户的价值提供
可判断的量化标准,否则要么因为无法量化评估而影响效率,要么因为评估失当引发不公平甚至导致失控。

其次,区块链尚处于发展阶段,区块链上大量数据及资产的价值等着去发现。有效的价值衡量体系将使得冰山下的部分得以露出,催生新兴应用甚至领域出现。比如区块链上的借贷和征信类服务、数据搜索和个性化推荐、原生跨链交易和数据交换等,价值衡量体系将使这些领域突破瓶颈。

最后,生态建设需要有效的激励和健康的发展方向。有效激励的基础就是有效的价值衡量体系。如果没有价值衡量体系,甚至价值衡量体系是歪曲的,那么就会导致激励机制失效,导致整个区块链系统不可避免地走向灭亡。

综上,一个区块链价值衡量标准需要具备三个特点:

\begin{itemize}
\item{真实性} 一个好的价值衡量标准应该能够准确反映出区块链经济系统的特征,这样才能在相应的领域具有
足够的公信力;
\item{公平性} 价值衡量标准为相应的激励提供了依据,因此,这一
依据必须足够公平,才能防止作弊或操纵带来的“劣币驱逐良币”现象;
\item{多样性} 需要使用数据及数字资产价值的场景可能是多种多样的,其使用方式及对应的激励方式不尽相同,
因此相应的价值衡量标准既不能脱离应用场景,亦要满足前述的真实性及公平性。
\end{itemize}

星云指数(Nebulas Rank)将是一个满足以上三个特点的区块链价值衡量标准。

为了体现真实性,我们参考了诸多指标,最终我们定义星云指数为:衡量账户地址对于区块链这一经济系统的贡献度。

本质上来说,区块链作为一个经济体,并不违背经典货币理论。
我们认为区块链系统之上的加密数字货币应该具备基本的货币属性,并且加密数字货币的价值源于其流通性。
因此,加密数字货币的交易记录是衡量加密数字货币这一经济体的有效数据来源。更进一步的,我们认为每个账户发
起的每一笔交易都在一定程度上增加了加密数字货币的流通性。微观角度看,每个账户的交易行为都最终反映在了区块链系统的价值中;
从宏观角度看,我们将所有账户地址的星云指数的和定义为整个区块链系统的经济总量。

为了验证星云指数设计的有效性,我们在基于以太坊的链上数据中计算了所有账户的星云指数之和,
并与Coinmarketcap.com中同期的以太坊市值进行了对比。我们的对比表明了二者具有很强的正相关性(0.85),即星云指数既能够在微观层面衡量每个账户
对经济系统的贡献,亦能在宏观层面反映整个经济系统总量的变化。

为了保证公平性,我们设计了能够有效抵抗操纵的计算函数。并论证出了星云指数在抵抗操纵方面可达到的性能。

在星云指数的理论基础上,为了满足多样性的需要,我们将星云指数分为核心星云指数(\nrcore)和计算基于核心星云指数的扩展星云指数(\nrext)两部分。

核心星云指数针对
区块链中不同账户对于区块链系统的贡献度给出了计算方法。其计算基于两个参考因素:其一,账户在一定时期内的资产中值;其二,账户在一定时期内
的出入度衡量。

扩展星云指数则基于核心星云指数来构建,针对区块链生态中各应用可能
需要的价值尺度给出了不同的计算方法,以便更符合不同场景的实际需要。并举了几种扩展星云指数的计算方法作为参考,例如:如何根据核心
星云指数对智能合约进行排名;如何将星云指数拓展到多个维并给予不同的权重等。

本黄皮书除了给出理论论证,还解决了几个星云指数落地时必须面对的问题,例如星云指数是否上链,星云指数的计算如何更新等。对星云指数实际落地给出了具体的工作方向。



\whitepaper{
特殊提示:
本星云指数黄皮书作为专项讨论星云指数的黄皮书,对星云技术白皮书(2018年4月发布的1.02版本)~\cite{Nabulas}中星云指数相关章节进行了大幅度的升级和拓展。
相对于一年前的概念论证,经过一年的深入思考与实际验证,我们有信心和能力设计出更为严谨的算法,并对星云指数的更多实际细节问题提供明确的解决方案或方向。
为了方便阅读,我们将使用实线框高亮解释技术白皮书提及过、并且在本黄皮书中有升级的相关技术点。
}
