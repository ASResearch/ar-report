% !TEX root = main.tex
\section{概要}
作为比特币的基石,区块链技术带来了真正意义上的“去中心化”交易平台。区块链具有可溯源性和匿名性的特点,由于其可溯源特性,以比特币、以太坊为代表的加密数字货币已逐渐具备智能资产属性。
其还具有匿名性,在一定程度上为链上的交易和数据提供了安全保障,
由于匿名性,我们无从推断多个属于不同账户的数据和资产是否属于同一个用户。这种关联性的丧失,从某种程度
上来说是区块链所追求的目标,但同时也造成了一些增值应用的缺失~\cite{meiklejohn2013fistful},为进一步发掘区块链上数据及资产价值带来了挑战。而许多应用均需要相应的价值衡量标准,例如,区块链上的借贷和
征信类服务、数据搜索和个性化推荐、原生跨链交易和数据交换等。


%然而,一个更加本质的问题是:如何定义区块链上的数据及资产的价值,这种价值又如何量化。
%我们认为,价值的量化是依赖于价值所使用的场景的。例如区块链上的去中心化应用推荐系统需要考量
%相应的数据的价值,以便作出更好的推荐,作为对比,跨链的数据交换也需要衡量相应的数据的价值,
%这两种场景下的对数据的价值的衡量,很难采用统一的标准。

在被价值衡量体系缺失困扰的同时,我们发现,通用的价值衡量标准很难建立。这一认识基于价值衡量标准需要具备如下三大特点:

%建立通用的价值衡量标准是十分困难的,这种主要体现在以下几个方面:
\begin{itemize}
\item{多样性} 需要使用数据及数字资产价值的场景可能是多种多样的,其使用方式及对应的激励方式,
都不尽相同,不能脱离应用场景建立通用的价值衡量标准;
\item{真实性} 一个好的价值衡量标准应该能够准确反映出至少某一方面的特征,有助于解决某一领域的问题,这样才能在相应的领域具有
足够的公信力;
\item{公平性} 这是价值衡量标准的根本所在。价值衡量标准为相应的激励提供了依据,因此,这一
依据必须足够公平,才能防止作弊或操纵带来的“劣币驱逐良币”现象。
\end{itemize}

由此引出一个更加本质和关键的问题:如何定义区块链上的数据及资产的价值?这种价值又如何被量化?

%我们认为,虽然一个通用的价值衡量标准很难建立,但是在区块链上量化数据或数字资产的方法是十分关键的,

我们试图在星云链的场景中给出一个建立价值尺度的方法,即星云指数(Nebulas Rank)。
星云指数对于如何建立一个有效的价值衡量尺度提供了完整的方法论支持,从价值衡量标准三大特点入手,解决了建立通用价值衡量标准的难题:

\begin{itemize}
\item{多样性} 星云指数包括核心星云指数(\nrcore)和计算基于核心星云指数的扩展星云指数(\nrext)两部分。其中,核心星云指数针对
星云链中不同账户对与星云链的贡献度给出了计算方法,此为基石;而扩展星云指数则针对星云链生态中各应用可能
需要的价值尺度给出了计算方法。满足不同应用和不同场景的需要,从而实现多样性;
\item{真实性} 核心星云指数代表了账户地址对于星云链的贡献度,经计算表明,核心星云指数
与星云链的价值呈现出了很强的正相关性,也就是说,真实地反映了星云链的价值。因此我们认为,核心星云
指数的真实性有保证,相应的,以核心星云指数为基石的扩展星云指数同样真实可鉴。
\item{公平性} 为了抵抗操纵,我们引入了全新的计算函数,并且被证明能够有效抵抗“女巫攻击”~\cite{cheng2005sybilproof}。%\sout{为了抵抗操纵,我们引入了博弈论,所设计的星云指数“效用函数”在纳什均衡下,不存在“操纵”
%的空间。}
\end{itemize}

\noindent 特殊提示:\\
本文的描述与星云技术白皮书(2018年4月发布的1.02版本)~\cite{Nabulas}中对星云指数的描述有所出入,这是因为经过一年的深入思考与验证,我们有信息和能力提出更为严谨的算法。为了加以区分和比较,我们对高亮了技术白皮书中不够严谨的章节,并将给出进一步的说明。
