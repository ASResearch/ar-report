% !TEX root = main.tex
\section{证明}
\label{sec:appendix_proof}
\subsection{特征\ref{prop:one}证明}
\begin{proof}
\label{proof:one}
对于任意$x_1>0$,$x_2>0$,有

\[
\begin{aligned}
f(x_1+x_2)&=\frac{x_1+x_2}{1+e^{a+b\cdot (x_1+x_2)}}\\
&=\frac{x_1}{1+e^{a+b\cdot (x_1+x_2)}}+\frac{x_2}{1+e^{a+b\cdot (x_1+x_2)}}\\
&=\frac{x_1}{1+e^{b\cdot x_2}\cdot e^{a+b\cdot {x_1}}}+\frac{x_2}{1+e^{b\cdot x_1}\cdot e^{a+b\cdot {x_2}}}\\
\end{aligned}
\]

在公式\ref{eq:rank-param}中,有$b<0$,因此$0<e^{b\cdot x_1}<1$,$0<e^{b\cdot x_2}<1$,进一步地,

\[
\frac{x_1}{1+e^{b\cdot x_2}\cdot e^{a+b\cdot {x_1}}}>\frac{x_1}{1+ e^{a+b\cdot {x_1}}}=f(x_1)
\]

\[
\frac{x_2}{1+e^{b\cdot x_1}\cdot e^{a+b\cdot {x_2}}}>\frac{x_2}{1+ e^{a+b\cdot {x_2}}}=f(x_2)
\]

即
\[
f(x_1+x_2)>f(x_1)+f(x_2)
\]
\end{proof}

\subsection{特征\ref{prop:two}证明}
\begin{proof}
对于任意$x_1>0$,$x_2>0$,有

\begin{equation}
\label{eq:fx_proof}
\begin{aligned}
f(x_1+x_2)-f(x_1)-f(x_2)&=\frac{x_1+x_2}{1+e^{a+b\cdot (x_1+x_2)}}-\frac{x_1}{1+e^{a+b\cdot x_1}}-\frac{x_2}{1+e^{a+b\cdot x_2}}\\
%&=\frac{x_1}{1+e^{a+b\cdot (x_1+x_2)}}+\frac{x_2}{1+e^{a+b\cdot (x_1+x_2)}}\\
&=(\frac{x_1}{1+e^{b\cdot x_2}\cdot e^{a+b\cdot {x_1}}}-\frac{x_1}{1+e^{a+b\cdot x_1}})+(\frac{x_2}{1+e^{b\cdot x_1}\cdot e^{a+b\cdot {x_2}}}-\frac{x_2}{1+e^{a+b\cdot x_2}})\\
\end{aligned}
\end{equation}

这里用函数$g(x_1,x_2)$表示左边部分,$h(x_1,x_2)$表示右边部分,即:

\begin{equation}
\label{eq:gx_func_proof}
g(x_1,x_2)=\frac{x_1}{1+e^{b\cdot x_2}\cdot e^{a+b\cdot {x_1}}}-\frac{x_1}{1+e^{a+b\cdot x_1}}
\end{equation}

\begin{equation}
\label{eq:hx_func_proof}
h(x_1,x_2)=\frac{x_2}{1+e^{b\cdot x_1}\cdot e^{a+b\cdot {x_2}}}-\frac{x_2}{1+e^{a+b\cdot x_2}}
\end{equation}

因此~\eqref{eq:fx_proof}对于$x_1$和$x_2$的极限可以表示为:
\[
\mathop {\lim }\limits_{\scriptstyle x_1 \to \infty  \hfill \atop  \scriptstyle x_2 \to \infty  \hfill}[f(x_1+x_2)-f(x_1)-f(x_2)]=\mathop {\lim }\limits_{\scriptstyle x_1 \to \infty  \hfill \atop  \scriptstyle x_2 \to \infty  \hfill}g(x_1,x_2)+\mathop {\lim }\limits_{\scriptstyle x_1 \to \infty  \hfill \atop  \scriptstyle x_2 \to \infty  \hfill}h(x_1,x_2)
\]

其中
\[
\begin{aligned}
g(x_1,x_2)&=\frac{x_1}{1+e^{b\cdot x_2}\cdot e^{a+b\cdot {x_1}}}-\frac{x_1}{1+e^{a+b\cdot x_1}}\\
&=\frac{x_1\cdot e^{a+b\cdot x_1}\cdot(1-e^{b\cdot x_2})}{(1+e^{b\cdot x_2}\cdot e^{a+b\cdot x_1})\cdot(1+e^{a+b\cdot x_1})}\\
&<\frac{x_1\cdot e^{a+b\cdot x_1}\cdot(1+e^{a+b\cdot x_1})}{(1+e^{b\cdot x_2}\cdot e^{a+b\cdot x_1})\cdot(1+e^{a+b\cdot x_1})}=\frac{x_1\cdot e^{a+b\cdot x_1}}{1+e^{b\cdot x_2}\cdot e^{a+b\cdot x_1}}\\
&<\frac{x_1\cdot e^{a+b\cdot x_1}}{1+e^{a+b\cdot x_1}}=\frac{x_1}{1+\frac{1}{e^{a+b\cdot x_1}}}
\end{aligned}
\]

对$\frac{x}{1+\frac{1}{e^{a+b\cdot x}}}$求极限,根据洛必达法则,
\[
\begin{aligned}
\lim_{x \to \infty}\frac{x}{1+\frac{1}{e^{a+b\cdot x}}}&=\lim_{x \to \infty}\frac{1}{(e^{-a-b\cdot x})'}\\
&=\lim_{x \to \infty}\frac{1}{-b\cdot e^{-a-b\cdot x}}\\
&=0
\end{aligned}
\]

根据~\ref{proof:one},有$g(x_1,x_2)>0$,因此根据夹逼定理有:

\[
\mathop {\lim }\limits_{\scriptstyle x_1 \to \infty  \hfill \atop  \scriptstyle x_2 \to \infty  \hfill}g(x_1,x_2)=0
\]

同理,可以求得:
\[
\mathop {\lim }\limits_{\scriptstyle x_1 \to \infty  \hfill \atop  \scriptstyle x_2 \to \infty  \hfill}h(x_1,x_2)=0
\]

因此,
\[
\mathop {\lim }\limits_{\scriptstyle x_1 \to \infty  \hfill \atop  \scriptstyle x_2 \to \infty  \hfill}[f(x_1+x_2)-f(x_1)-f(x_2)]=0
\]


\end{proof}