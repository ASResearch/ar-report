% !TEX root = main.tex

\section{背景及相关技术}
\subsection{讨论}
主要参考nem白皮书,背景:缺少价值尺度
pos:stake
pow:computation power
poa:
poi:

neb:用statke,鼓励资产流通,需要引入流通性,but 容易受到sybil attack
现有算法 e.g. page rank

我们的工作基于前人的工作

\subsection{区块链发展情况}
2008年10月,中本聪(Satoshi Nakamoto)公开发表比特币白皮书~\cite{Nakamoto2008}。比特币作为区块链的太初应用,践行了其作为“一个去中心化电子现金系统”的初衷。 比特币的产生不依赖于任何机构,而是根据特定算法,依靠大量计算产生,保证了比特币网络分布式记账系统的一致性。

通过特定的脚本语言,我们可以利用比特币实现第三方支付交易、高效小额支付(efficient micro-payments)等功能。

然而比特币脚本语言的设计存在很多缺陷,支持较少指令,并不符合图灵计算标准。但随着区块链技术的不断深入研究,涌现了更多后继者,例如以太坊Ehereum~\cite{buterin2013ethereum}致力于实提供图灵完备的智能合约(Smart Contracts)。

智能合约是比特币系统里可以用技术手段来强制执行的合约,基于智能合约,人们可以开发可以实现复杂功能的分布式应用(DApps),

随着DApps的兴起,在诸多交易以及智能合约的调用过程中,隐藏着难以估计的信息。后者相比传统交易数据,往往具有更多维度,因此无法使用传统价值衡量标准评估。



%现在链超多,跨链需要衡量价值。。。


\subsection{图中节点的排名算法}
由于智能合约的引入,当前以以太坊为代表的新一代区块链项目,不仅仅是电子货币交易平台,而是在此之上建立了复杂庞大的经济体系。尚不存在一个合理的方式去评估链上实体(例如用户地址)的价值,例如我们需要知道:哪些实体为整个区块链生态贡献较大,又应如何衡量这类贡献。

作为Google早期的核心算法,PageRank设计初衷是用于解决链接分析中网页排名问题,随着国内外学者的深入研究,PageRank算法被广泛应用于其他方面,例如学术论文的重要性排名~\cite{},网络爬虫~\cite{},关键词与句子~\cite{}的抽取,以及基于PageRank的社交用户的影响力排名研究。

此后,涌现出一些其他在PageRank算法基础上进行改进的研究,其中较为著名的包括LeaderRank算法,其最初应用于在线社交网站的用户排名。LeaderRank算法的基本思想是在整个网络已有节点外另外增加一个背景节点(Ground Node),并且将其与已有所有节点建立双向连接,使得N+1个节点的新网络成为一个强连通网络,在此基础上基于类似PageRank算法计算可以得到原有N个节点的“重要性”排序。

上述传统排名算法,在区块链上都存在不同问题。例如无法应对女巫攻击(Sybil Attack),



\subsection{对抗操纵}
paper:

博弈论相关工作了解一下

看着还行,但是假设太强,抗操纵性不行的。
