\section{区块链经济模型}
区块链之上的加密数字货币无疑正在成为一种新的经济体~\cite{neweco},这一经济体将包含越来越多的行为及个体,整个系统也逐渐变得更加复杂,由最初的、单纯的数字货币系统,正在逐步成为智能合约、甚至应用的基础平台。

尽管区块链之上的加密数字货币是一种新型的经济体,我们认为,其仍然遵循必然的经济规律。也就是说,数字货币对应的经济体的变化仍然遵循经济学中的客观规律,
使用传统的、严谨的经济学方法对加密数字货币进行研究依然是有效的。

我们希望Nebulas能够成为一个高效的经济体,既,Nebulas能够被大家使用,用于各种场景下的交易、并支撑各种智能合约及应用。
因此,我们需要深入的理解一个经济体是如何变得高效的,以及如何定量的描述这些指标。

在传统的经济学中,存在很多指标用于描述经济系统,例如``流动性'',``流通性'',``货币流动性''等。
我们认为使用``货币流通速度''是一个便于使用的指标,Selden~\cite{selden}指出,货币流通速度是一个时期中货币的流量与该时期
中货币平均存量的比率。也就是说xxxx。

{\textit 在技术白皮书中,我们使用了``流动性''一词,然而,``流动性''一词缺乏严格的定义,即使在经济学中,这一涵义也是十分广泛的。例如,在《新帕尔格雷夫金融学辞典》中,对流
动性(liquidity) 解释的专门词条包括了完全不同的三个方面。兰德尔·克罗兹勒~\cite{randall}指出 ,在过去六个月里 ,有 2795 篇
独立的文章谈到了流动性 ,但流动性是什么含义 ,大概有 2795 种不同的说法。}



我们认为在经济系统中的排名,意味着一种激励的方向,这个方向应该与经济系统增长的方向保持一致。

更进一步的,我们认为,经济系统中的排名应该反应其对经济系统的贡献

更加形式化的,f(a+b) > f(a) + f(b) 及 f(a+b) < f(a) + f(b)
