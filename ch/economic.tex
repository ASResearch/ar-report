% !TEX root = main.tex
\section{区块链经济模型}


\subsection{加密数字货币表示}
作为加密数字货币,其与传统经济体的一个重要差别在于所有交易都是可追踪的,这为我们定量研究每一笔交易对整个经济系统的
影响提供了强力的数据支持。

一般地,一个加密数字货币系统可以描述为一个二元组$(\mathcal{L}, \mathcal{U})$,其中$\mathcal{L}$为账本系统,$\mathcal{U}$为
数字货币的用户集合。账本系统$\mathcal{L}$可以被描述为一个三元组,即
\begin{align}
\mathcal{L} = (\mathcal{A}, \mathcal{D}, \mathcal{T})
\end{align}

\noindent 其中$\mathcal{A}$ 为账户的集合,$\mathcal{D}$为初始状态下各个账户余额的集合,$\mathcal{T}$为交易记录的集合,每条交易
记录为一个四元组,即

\begin{align}
\mathcal{D} = \{a \rightarrow d, a{\in}\mathcal{A}, d{\in}R^*\}
\end{align}
\begin{align}
\mathcal{T} = \{(s, t, w, \tau)\}
\end{align}

\noindent 其中$s$为交易的发起地址,$t$为交易的目的地址,$w$为交易的金额,$\tau$为交易的时间。

一个账户受控于相应的用户,该用户能够发起交易,我们记为

\begin{align}
u \dom a. \quad u\in \mathcal{U}, a\in \mathcal{A}
\end{align}

\noindent 一个用户可以控制多个账户,即

\begin{align}
A(u) = \{\forall a\in \mathcal{A} : u \dom a\}
\end{align}

\noindent 而一个账户仅能被一个用户所控制,即

\begin{align}
\forall u_1, u_2 \in \mathcal{U} : A(u_1) \cap A(u_2) = \phi
\end{align}

需要注意的是,上述模型是任意加密数字货币系统的合理简化。我们在模型中未区分链上数据或链下数据、未引入成交价格、未引入智能合约的调用等。
此外有一个特殊情况是中心化交易所的账户。通常来说,一个交易所账户会被分配给多个用户,每一个用户使用不同的地址进行交易,在交易所内的交易由交易所在中心化的数据库中进行记录,而不会记录在链上。
这就意味着我们不能仅仅通过链上数据获取交易所内的交易记录。但是,可以在交易所的配合下获取相应的数据,进一步将交易所账户映射为不同的多个账户,从而使用上述的模型进行描述。



\subsection{加密数字货币模型}
尽管数字货币和传统的商品货币以及法币存在较大区别,但经典的货币理论仍然可以提供指导价值以及借鉴作用。加密数字货币作为新型经济体的通行货币~\cite{swan2015blockchain},承载了货币的属性,具有三种职能:交换媒介、计价单位和价值储藏手段。


在此我们建立一个简单且经典的货币模型来帮助理解星云指数的物理意义。

首先,我们需要给出加密数字货币生态系统中衡量“流通性”的指标。需要加以区别的是,经济学中经常出现的另一个概念:流动性(liquidity),
用以说明一种资产兑换为交换媒介的容易程度。由于货币本身就是是经济中的交换媒介,所以货币是最具流动性的资产。

{\color{red} 在星云技术白皮书~\cite{Nabulas}中,我们使用了``流动性''一词。然而,``流动性''一词缺乏严格的定义,即使在经济学中,这一涵义也是十分广泛的。
例如,在《新帕尔格雷夫金融学辞典》中,对流
动性解释的专门词条包括了完全不同的三个方面。兰德尔·克罗兹勒~\cite{randall}指出,在过去六个月里,有2795篇
独立的文章谈到了流动性,但流动性是什么含义,大概有2795种不同的说法。在本黄皮书中的“流动性”,统一特指货币流通速度(velocity of money),即指单位货币在一定时期内的周转(或实现交换)次数。}

我们用货币流通速度~\cite{selden}表示加密数字货币的周转速率,即单位数字货币在一定时期内(在本文中为一天)的周转次数,用$V$表示。根据经典的货币数量论,数量方程式表示如下:

\begin{align}
M\times V=P\times Y
\label{eq:currency}
\end{align}

其中$M$表示整个经济系统中的货币数量,$V$是货币流通速度,$P$表示物价水平(用每单位经济产量的货币单位来衡量,所以货币价格是$\frac{1}{P}$),$Y$表示真实经济产量(真实GDP)。该方程式说明,货币数量乘以货币流通速度等于产品的价格乘以产量。

%统计表明,在健康的经济系统中,货币流通速度是较为稳定的。
%以美国经济发展情况为例,最近六十年美国名义供给与名义GDP都增加了30倍左右,但货币流通速度变化并不大。因此在数字货币经济体系中,货币流通速度基本不变的假设是符合现实环境的。

对于货币总量$M$, 星云链类似以太坊,星云生态环境中总货币量保持着稳定增长(目前星云币的增发比例暂定为4\%)。不同于比特币,后者的发行货币总量最终将稳定在2100万。
货币流通速度$V$可以描述为一段时间内流通的货币总量与当时货币总量的比值。
因此式~\ref{eq:currency}可以进一步写为
\begin{align}
(M + \Delta{m}) \times \frac{\sum_{(s, t, w, \tau)\in \mathcal{T}}{w}}{M} = P \times Y
\label{eq:cur_ext}
\end{align}
\noindent 其中,$\Delta{m}$为增发的货币量。


%上述模型并非完整的市场模型,例如此处并未考虑投资者心理活动,后者会影响货币需求。为保证模型不至于过于复杂,这里暂不讨论。

对于物价水平$P$,无论是古典货币理论还是新凯恩斯理论,都始终认同货币供给与需求最终决定货币价值的观点。从长期来看,物价总水平会调整到使货币需求等于货币供给的水平。

然而短期内,物价总水平本身并不能使货币供求平衡。在健康的经济系统中,物价水平增长速度往往小于货币总量的增长速度。通过适当增加货币供给(也可描述为降低利率),在物价$P$增长的同时也保证物品与服务需求量$Y$的增加。另一方面,同时需要对物价水平的增速进行一定的控制,以保证用户不会因为过强的增长预期而保持持有数字货币的状态,从而导致流通速度的降低。

%保持物价$P$的增长,以应对债券市场对投资的预期{\color{red} 需要确认};



对于真实经济产量$Y$,经济学家通常用真实GDP表示,即“某一既定时期一个国家生产的所有最终物品与服务的市场价值”。我们认为加密数字货币的价值来源与其流通性,即每一笔流通都在一定程度上为整个经济总量做了贡献。也就是说,当一笔实际的交易发生时,既在一定程度上增加了加密数字货币的流通性,也在一定程度上增强了人们对数字货币的认可及信心。
因此,我们认为式~\ref{eq:cur_ext}中的$Y$来源与每一笔交易,考虑到经济系统的主体为账户,也可以认为$Y$来源于各个账户上发生的交易,即

\begin{align}
Y=\sum_{a\in \mathcal{A}} \mathcal{C}(a)
\end{align}
\noindent 其中$\mathcal{C}(a)$表示账户$a$对整个经济产量的贡献,即星云指数。

数字货币的发展依赖于社区的发展,因此,我们认为量化社区中每个账户对于经济总量的贡献,为正确的激励提供了必要的基础。
基于此,经济系统可以对账户产生明确的激励(例如星云技术白皮书中的贡献度证明机制(Proof of Devotion, PoD),
亦可能产生不确定的激励(例如,搜索引擎中由于星云指数的影响而对搜索结果的排序产生的影响)。
不同于传统货币理论,在加密数字货币中,直接的、原生的激励是增发货币的发放。



%需要注意的是,当用户减少交易频率而更倾向于长期持有星云币时,市场流通星云币则会减少,星云币价格也会对应升高,但其流动性会降低。
%这种变化反映出星云币的投机资产特性,而削弱了其交易特性。
%这也是目前比特币等通缩货币所面临的主要问题,经济学称之为“大国会山托儿合作社危机”(The Great Capitol Hill Baby Sitting Co-op Crisis)~\cite{hens2007great},在此暂时不做详细讨论。

\begin{comment}

\subsection{经济激励模型}
完整的讨论经济激励模型,超出了本文的范围,然而,考虑到星云指数可能产生的激励效应,有必要对星云指数之后的经济激励给出必要的说明。

星云指数衡量了一个账户对经济系统的贡献,基于此,经济系统可以对账户产生明确的激励(例如技术白皮书中的贡献度证明机制(Proof of Devotion, PoD),
亦可能产生不确定的激励(例如,搜索引擎中由于星云指数的影响而对搜索结果的排序产生的影响)。在加密数字货币中,直接的、原生的激励是增发货币的发放。

我们认为,加密数字货币的最根本的价值来源于流通,即在更多的交易中产生更多的价值

根据式~\ref{eq:currency}

当社区用户产生价值增幅大于星云币增发幅度时,星云币会对应升值或者星云币流通性增加,即用户交易频率增加,从而达到新的平衡,反之亦然。
因此在计算星云经济系统的产出时,我们采用星云指数作为衡量指标,同时我们希望该指标能够促进星云币流通速度的增加。

星云指数的物理意义是{\textbf 用户对经济系统增长的贡献}。更进一步的,所有用户的星云指数的总和,应该与经济体的整体价值正相关{\color{red} xxx}.
\end{comment}
