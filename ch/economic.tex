% !TEX root = main.tex
\section{区块链经济模型}


\subsection{加密数字货币系统}
作为加密数字货币,其与传统经济体的一个重要差别在于所有交易都是可追踪的,这为我们定量的研究每一笔交易对整个经济系统的
影响提供了强力的数据支持。

一般地,一个加密数字货币系统可以描述为一个二元组$(\mathcal{L}, \mathcal{U})$,其中$\mathcal{L}$为账本系统,$\mathcal{U}$为
数字货币的用户集合。更进一步的,账本系统可以描述为一个三元组,即
\begin{align}
\mathcal{L} = (\mathcal{A}, \mathcal{D}, \mathcal{T})
\end{align}
\noindent 其中$\mathcal{A}$ 为账户的集合,$\mathcal{D}$为初始状态下各个账户的余额的集合,$\mathcal{T}$为交易记录的集合,每条交易
记录为一个四元组,即
\begin{align}
\mathcal{D} = \{a \rightarrow d, a{\in}\mathcal{A}, d{\in}R^*\}
\end{align}
\begin{align}
\mathcal{T} = \{(s, t, w, \tau)\}
\end{align}
\noindent 其中$s$为交易的发起地址,$t$为交易的目的地址,$w$为交易的金额,$\tau$为交易的时间。

任意一个账户被相应的用户控制,能够发起交易,我们记为
\begin{align}
u \dom a. \quad u\in \mathcal{U}, a\in \mathcal{A}
\end{align}
\noindent 一个用户可以用户多个账户,即
\begin{align}
A(u) = \{\forall a\in \mathcal{A} : u \dom a\}
\end{align}
\noindent 更进一步的,一个账户仅能被一个用户所控制,即
\begin{align}
\forall u_1, u_2 \in \mathcal{U} : A(u_1) \cap A(u_2) = \phi
\end{align}

需要注意的是,上述模型是任意加密数字货币系统合理简化,例如,我们在模型中未区分链上数据或链下数据、未因为成交价格、未引入智能合约的调用等。
中心化的交易所账户是特别的。通常来说,一个交易所账户会被分配给多个用户,每一个用户使用不同的地址进行交易,在交易所内的交易,由交易所在中心化的数据库中进行记录,而不会记录在链上,
这就意味着我们不能仅仅通过链上数据获取交易所内的交易记录。然而,可以在交易所的配合下获取相应的数据,进一步将交易所账户映射为不同的多个账户,从而使用上述的模型进行描述。



\subsection{经济模型}

作为新型的经济体~\cite{neweco},星云首先承载了作为数字货币的属性。尽管数字货币和传统的商品货币以及法币存在较大区别,但经典的货币理论仍然可以提供指导价值以及借鉴作用。
在此我们建立一个简单且经典的货币模型来帮助理解星云指数的物理意义。

首先我们给出星云生态系统中,衡量“流通性”的指标。需要指出的是,经济学中经常出现另一个概念:流动性(liquidity),
后者用来说明一种资产兑换为经济中的交换媒介的容易程度,由于货币是经济中的交换媒介,所以货币本身是最具流动性的资产。
在技术白皮书中提到的“流动性”,我们在这里统一特指货币流通速度(velocity of money),即在经济中货币易手的速度。

{\color{gray} 在技术白皮书中,我们使用了``流动性''一词,然而,``流动性''一词缺乏严格的定义,即使在经济学中,这一涵义也是十分广泛的。
例如,在《新帕尔格雷夫金融学辞典》中,对流
动性(liquidity) 解释的专门词条包括了完全不同的三个方面。兰德尔·克罗兹勒~\cite{randall}指出,在过去六个月里,有2795篇
独立的文章谈到了流动性,但流动性是什么含义,大概有2795种不同的说法。}


我们用货币流通速度~\cite{selden}表示星云币的周转速率,即单位星云币的周转次数,用$V$表示。根据经典的货币数量论,数量方程式表示如下:

\begin{align}
MV=PY
\end{align}

其中$P$表示物价水平(GDP平减指数),$Y$表示真实产量(真实GDP),$M$表示经济中的货币数量,而$V$则是上述流通速度。该方程式说明,货币数量乘以货币流通速度等于产品的价格乘以产量。

该数量方程式说明,经济中货币量的增加必然反映在其他三个变量中的一个上:物价水平的必然上升(货币贬值),产量必然上升,或货币流通速度必然下降。

不同于比特币发行货币总量固定,现有星云生态环境中总货币量采取定期增发的策略(目前星云币增发比例暂定为4\%)。


星云链依赖于社区的发展,在Nebulas生态环境中,我们希望社区贡献以及用户价值能够真实反应出经济产量,即:

\begin{align}
Y=\sum_{a\in \mathcal{A}} \mathcal{C}(a)
\end{align}
\noindent 其中$\mathcal(a)$表示账户$a$对整个经济产量的贡献,即星云指数。
在传统经济系统中,货币流通速度是较为稳定的。以美国经济发展情况为例,最近六十年美国名义供给与名义GDP都增加了30倍左右,但货币流通速度变化并不大。因此在星云经济体系中,货币流通速度基本不变的假设是符合现实环境的。

当社区用户产生价值增幅大于星云币增发幅度时,星云币会对应升值或者星云币流通性增加,即用户交易频率增加,从而达到新的平衡,反之亦然。
因此在计算星云经济系统的产出时,我们采用星云指数作为衡量指标,同时我们希望该指标能够促进星云币流通速度的增加。

需要注意的是,当用户减少交易频率而更倾向于长期持有星云币时,市场流通星云币则会减少,星云币价格也会对应升高,但其流动性会降低。这种变化反映出星云币的投机资产特性,而削弱了其交易特性。这也是目前比特币等通缩货币所面临的主要问题,经济学称之为“大国会山托儿合作社危机”(The Great Capitol Hill Baby Sitting Co-op Crisis)~\cite{hens2007great},在此暂时不做详细讨论。


\subsection{激励模型}
完整的讨论激励模型,超出了本文的范围,然而,仍然有必要讨论一下

星云指数的物理意义是{\textbf 用户对经济系统增长的贡献}。更进一步的,所有用户的星云指数的总和,应该与经济体的整体价值正相关{\color{red} xxx}.
