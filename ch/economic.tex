% !TEX root = main.tex

\section{区块链经济模型}
区块链之上的加密数字货币无疑正在成为一种新的经济体~\cite{neweco},这一经济体将包含越来越多的行为及个体,整个系统也逐渐变得更加复杂,由最初的、单纯的数字货币系统,正在逐步成为智能合约、甚至应用的基础平台。

尽管区块链之上的加密数字货币是一种新型的经济体,我们认为,其仍然遵循必然的经济规律。也就是说,数字货币对应的经济体的变化仍然遵循经济学中的客观规律,
使用传统的、严谨的经济学方法对加密数字货币进行研究依然是有效的。

我们希望Nebulas能够成为一个高效的经济体,既,Nebulas能够被大家使用,用于各种场景下的交易、并支撑各种智能合约及应用。
因此,我们需要深入的理解一个经济体是如何变得高效的,以及如何定量的描述这些指标。

在传统的经济学中,存在很多指标用于描述经济系统,例如``流动性'',``流通性'',``货币流动性''等。
我们认为使用``货币流通速度''是一个便于使用的指标,Selden~\cite{selden}指出,货币流通速度是一个时期中货币的流量与该时期
中货币平均存量的比率。也就是说xxxx。




{\color{gray} 在技术白皮书中,我们使用了``流动性''一词,然而,``流动性''一词缺乏严格的定义,即使在经济学中,这一涵义也是十分广泛的。
例如,在《新帕尔格雷夫金融学辞典》中,对流
动性(liquidity) 解释的专门词条包括了完全不同的三个方面。兰德尔·克罗兹勒~\cite{randall}指出,在过去六个月里,有2795篇
独立的文章谈到了流动性,但流动性是什么含义,大概有2795种不同的说法。}

\subsection{宏观模型}
作为新型的经济体,Nebulas首先承载了数字货币流通的平台。尽管数字货币和传统的商品货币以及法币存在较大区别,但经典的货币理论仍然可以提供指导价值以及借鉴作用。
在此我们建立一个简单且经典的货币模型来帮助理解Nebulas Rank的重要性。

首先我们给出星云生态系统中,衡量“流通性”的指标。需要指出的是,经济学中经常出现另一个概念:流动性(liquidity),后者用来说明一种资产兑换为经济中的交换媒介的容易程度,由于货币是经济中的交换媒介,所以货币本身是最具流动性的资产。在技术白皮书中提到的“流动性”,我们在这里统一特指货币流通速度(velocity of money),即在经济中货币易手的速度。


我们用流通速度表示星云币的周转速率,即单位星云币的周转次数,用$V$表示。根据经典的货币数量论,数量方程式表示如下:

\begin{align}
MV=PY
\end{align}

其中$P$表示物价水平(GDP平减指数),$Y$表示真实产量(真实GDP),$M$表示经济中的货币数量,而$V$则是上述流通速度。该方程式说明,货币数量乘以货币流通速度等于产品的价格乘以产量。

该数量方程式说明,经济中货币量的增加必然反映在其他三个变量中的一个上:物价水平的必然上升(货币贬值),产量必然上升,或货币流通速度必然下降。

不同于比特币发行货币总量固定,现有星云生态环境中总货币量采取定期增发的策略(目前星云币增发比例暂定为4\%)。


星云链依赖于社区的发展,在Nebulas生态环境中,我们希望社区贡献以及用户价值能够真实反应出经济产量,即:

\begin{align}
Y=\sum_i q(i)
\end{align}

在传统经济系统中,货币流通速度是较为稳定的。以美国经济发展情况为例,最近六十年美国名义供给与名义GDP都增加了30倍左右,但货币流通速度变化并不大。因此在星云经济体系中,货币流通速度基本不变的假设是符合现实环境的。

当社区用户产生价值增幅大于星云币增发幅度时,星云币会对应升值或者星云币流通性增加,即用户交易频率增加,从而达到新的平衡,反之亦然。因此在计算星云经济系统的产出时,我们采用Nebulas Rank作为衡量指标,同时我们希望该指标能够促进星云币流通速度的增加。

%\begin{itemize}
%\item{T:市场中所有参与者使用星云币进行支付的总交易量,用每秒钟发生的交易量来计量,这里采用美元作为单位方便后面统一价值;}
%\item{D:星云币用于支付时,暂时退出流通体系的时间跨度,单位为秒;}
%\item{S:市场上流通的星云币总量,其值等于星云币总量减去人们打算长期持有的星云币数量;}
%\item{P:星云币价格(对美元);}
%在供应侧,我们可以计算出每秒有$S/D$个星云币重新进入流通体系,而在需求侧,每秒钟所需用于支付交易的星云币数量为交易规模T(美元)乘以1美元对应$1/P$个星云币,即$T/P$。当达到供需平衡时,我们可以得到以下公式:
%\begin{align}
%S/D=T/P
%\end{align}

%从中推出星云币价格:
%\begin{align}
%P=\frac{TD}{S}
%\end{align}

%通常情况下D稳定(交易过程中星云币退出市场的持续时间不变),假设S不变或者变化缓慢(考虑到目前星云币增发比例暂定为4\%,市场流通总量较为稳定),那么星云币价格P和交易规模T成正比,即当交易需求增加,对应星云币价格也会增加。尽管上述模型较为简单,并未考虑投资者预期等复杂因素,但根据此模型已经能够分析得出货币价值与市场交易需求正相关的结论,这也与我们普遍认知一致。

需要注意的是,当用户减少交易频率而更倾向于长期持有星云币时,市场流通星云币则会减少,星云币价格也会对应升高,但其流动性会降低。这种变化反映出星云币的投机资产特性,而削弱了其交易特性。这也是目前比特币等通缩货币所面临的主要问题,经济学称之为“大国会山托儿合作社危机”(The Great Capitol Hill Baby Sitting Co-op Crisis)~\cite{hens2007great},在此暂时不做详细讨论。

\subsection{另一个版本}
PS:这里尝试从微观经济学来给出解释模型。

假设星云经济系统中,存在共计$N$个用户(或者称之为参与者,下文中两者表示同一概念)。用户$i$在任意时刻$t$需要发起数量为$x_{it}$的交易,后者$x_{it}$服从伯努利分布$B(1,p)$。对于用户$i$而言,其使用星云币进行交易的货币收益表示为$\frac{\widetilde{e}_{t+1}}{e}x_{it}$,其中$e$表示汇率,$\widetilde{e}_{t+1}$表示在t时刻星云经济系统仍然运行情况下的汇率,后者服从某个分布$G_{t+1}$。


在任意时刻$t$用户$i$选择策略$a_{it}\in \{T,\widehat{T}\}$,即使用星云币进行交易(表示为$T$),以及其他选择(例如通过其他交易渠道进行交易,表示为$\widehat{T}$)。在经济学中,每个参与者在参与博弈时采取不同行动策略会获取不同的收益,后者可以用收益函数(Payoff function)来表示。

综上,用户$i$交易行为的收益函数可以表示为:

\begin{align}
u(T;x_{it},e,\kappa_t,c_i)=\kappa_t \cdot \mathbb{E}_{it}v(\frac{\widetilde{e}_{t+1}}{e}x_{it})-c_i
\end{align}

其中$\kappa_t$表示在时刻$t$星云经济系统稳定的概率(该概率随时间增加而不断趋近于1,{\color{gray}相关证明后续需要的话会补充}),$c_i$表示用户交易产生的损失(例如额外的时间开销等),该损失存在上下限,因此对于任何用户$i$,有$c_i\in[c^-,c^+]$。
$\mathbb{E}_{it}$表示在给定环境下时刻$t$参与者$i$效用函数(utility function)的期望,需要指出的是,所有参与者均假设为风险厌恶倾向,因此用户效用函数用弱凸函数(weakly concave function)$v$表示,即效用随着货币收益增加而增加,但增加率递减。

对应用户$i$采取其他策略的收益函数表示为$u(\widehat{T};x_{it})$.

这里用$a_{it}$表示用户$i$在$t$时刻的决策,当$u(T;x_{it},e,\kappa_t,c_i)\geq u(\widehat{T};x_{it})$时,用户$i$选择采用星云币完成交易,即$a_{it}=T$,反之亦然。


因此市场需求可以表示为所有用户因为交易产生的星云币总量:
\begin{align}
\label{demandfunc}
Q_t(e;x_t)=\sum_ix_{it}1\{a_{it}=T\}=\sum_i x_{it}1\{u(T;x_{it},e,\kappa_t,c_i)\geq u(\widehat{T};x_{it})\}
\end{align}

市场供应量表示为常数$\bar{B}$,{\color{gray} 类似于上面所说$S/D$不变或者变化缓慢}

当达到市场均衡时,星云币的价格满足供给量与需求量相等。此时的经济系统满足如下特性:

1. 根据市场结清理论(market clearing),$e_t$是当需求小于供给时的最低汇率:

\begin{align}
\label{etfunc}
e_t=min\{e:Q_t(e;x_t)\leq\bar{B} e \}
\end{align}

2. 根据理性预期理论(rational expectations),在时刻$t$,参与者会充分利用所得信息来预期下一阶段的汇率分布$G_{t+1}$,并且该预期可被认为是合理正确的,这里用$\mu_{it}$表示用户$i$的理性预期。
\begin{align}
\label{me_func}
\mu_{it}=G_{t+1}
\end{align}



定理: 存在且唯一的汇率$e^{*}(x_t)$使得供需达到平衡.


证明:

首先在某个时刻$t>\hat{t}$,假设存在所有用户都选择使用星云币发起交易,我们可以找到满足此情况的一个充分必要条件:
\begin{align}
\mathbb{E}v(\frac{\widetilde{Z}}{N})-c^+>u(\widehat{T};1)
\end{align}

其中$\widetilde{Z}$是服从二项分布$Bi(N,p)$的随机变量。



汇率$e_t$则会趋近于:

\begin{align}
e_t=\sum_i x_{i,t} / \bar{B}
\end{align}

此时收益函数可以表示为:
\begin{align}
u(T;1,e,\kappa_t,c_i)=\kappa_t \cdot \mathbb{E}_{it}v(\frac{\widetilde{e}_{t+1}}{e})-c_i\geq \kappa_t \mathbb{E}_{it}v(\frac{\widetilde{e}_{t+1}}{N/\bar{B}})-c^+ 
\approx \kappa_t \mathbb{E}v(\frac{\widetilde{Z}}{N})-c^+>u(\widehat{T};1)
\end{align}

即此时对于所有参与者,使用星云币交易为最优选择。

根据上文,$G_{t+1}$表示$\widetilde{e}_{t+1}$的分布,在$\hat{t}$时刻,所有用户参与并使用星云币进行交易,因此$\widetilde{e}_{\hat{t}+1}$已知并且近似于$\widetilde{Z}/\bar{B}$的分布,即已知$G_{t+1}$,同理可以推出已知$G_t$。基于理性预期,市场需求可以根据公式\eqref{demandfunc}对每个用户累加求和得出。公式\eqref{demandfunc}为连续函数,且当$e$从0逐渐增大时,$Q_t(e;x_t)$从$\sum_ix_{it}$减少到趋近于0.这意味着,一定存在某个$e_t$满足公式\eqref{etfunc},即存在汇率$e_t$满足供需平衡。
$\square$






\subsection{交易激励}

我们认为在经济系统中的排名,意味着一种激励的方向,这个方向应该与经济系统增长的方向保持一致。为了保证或者进一步提升经济系统的货币的价值,根据上述模型,需要增加市场的交易规模。对应地,在Nebulas经济系统中,需要激励用户的交易行为。

我们将贡献度(Salience)作为衡量用户重要性指标之一,更进一步的,我们认为,经济系统中的排名应该反应其对经济系统的贡献。因此Nebulas Rank将

%更加形式化的,f(a+b) > f(a) + f(b) 及 f(a+b) < f(a) + f(b)
