\section{核心星云指数的实现}
核心星云指数的完整实现不在本文的讨论范围,此处我们仅讨论对于核心星云指数的实现方面的一些关键问题。

\subsection{是否上链?}
核心星云指数给出了每个账户对经济总量的贡献度,正常来说,每个节点都可以根据历史区块信息计算指定账户对经济总量的贡献度。
然而,我们是否需要周期性的将一定间隔时间内的星云指数上链呢?

我们认为并不需要将星云指数上链,这出于以下两个方面的考虑:
\begin{itemize}
\item 链上不适合存储如此大量的数据,即使对于IPFS,Genaro之类的以数据存储为目标场景的公链,
也不适合周期性的存储所有账户的核心星云指数;
\item 对于核心星云指数的计算会影响出块速度,核心星云指数的计算复杂度较高,如果将计算结果上链,
将很大程度上影响出块及验证速度,导致整个系统的TPS降低。
\end{itemize}
\noindent 综上,我们认为每个节点可以根据需要自行计算核心星云指数。

然而,在节点自行计算核心星云指数的情况下,一个重要的问题是如何保证核心星云指数计算的可信性。
例如,节点任意篡改核心星云指数的计算结果,从而根据错误的核心星云指数给出指定的激励。对于关键性的
应用,我们认为需要在各个节点对涉及核心星云指数的计算结果进行校验,以保证结果的公平性;
而对于非关键性的应用,对于核心星云指数的使用依赖于应用本身,是否对核心星云指数进行校验取决于应用。

在节点自行计算核心星云指数的情况下,另一个重要的问题是节点考虑到能耗问题,而拒绝计算核心星云指数。
对此,我们认为可以引入可信的核心星云指数服务,避免各个节点上的重复计算,对于该服务的使用,可以免费使用,
也使用次数计费。具体的实现及服务细节不在本文讨论范围内。

\subsection{核心星云指数的更新}
我们深知,核心星云指数是和加密数字货币的生态紧密相关的,随着生态的不管变化,核心星云指数的计算也是需要
不断更新的,尤其是其中的各个参数。因此,如何能够快速的更新核心星云指数的计算,是十分重要的,我们
通过星云原力,保证对核心星云指数计算的更新迭代。

我们会更新区块结构,新的区块结构中将包含核心星云指数的算法及参数(以LLVM IR形式),
星云虚拟机(NVM)作为算法的执行引擎,从区块中获得核心星云指数的算法及参数,并执行算法,在节点内获得账户的核心星云指数。

在算法或参数需要更新时,我们将和社区一起在新的区块中包含最新的算法及参数,
从而保证整个更新过程的及时性及平滑性,亦避免了可能到来的分叉。

%\subsection{分叉}
%我们预见到,不同的时期、不同的群体可能会有不同的利益诉求,这就意味这不同的核心星云指数算法的选择。
%我们认为随着整个社区的不断壮大,

%\begin{figure}
\centering
\begin{tikzpicture}
\pgfmathsetmacro{\XTD}{1.8}
\pgfmathsetmacro{\XMD}{1.2}
\pgfmathsetmacro{\YTD}{3.8}


\tikzset{
oblock/.style={draw, rectangle, on grid, align=center, minimum height=2ex},
nblock/.style={draw, rectangle, on grid, align=center, minimum height=2ex},
  node/.style={draw, circle, on grid, align=center, minimum height=2ex},
  thread/.style={draw, rectangle, on grid, align=center,color=gray!30,
  fill=gray!30,
  rounded corners,
  minimum height=3ex,fit=#1},
}

\node[oblock, fill=gray!30] (g) at (0, 0){$Genesis$};
\node[oblock] (v1) at (\XTD, 0) {$B_1$};
\node at (1.5*\XTD, 0) {$\dots$};
\node[oblock] at (2.5*\XTD, 0) {$Block_i$};
\node[nblock] at (3.5*\XTD, 0) {$\begin{aligned}
     \mathcal{C}(v, a, b, c, d, \theta, \mu, \lambda)
  \end{aligned}$};

\node at (4*\XTD, 0) {$\dots$};
\node[oblock] at (6*\XTD, 0) {$Block_i$};



\end{tikzpicture}
\label{fig:loop}
\caption{交易图中的交易环}
\end{figure}

