
\begin{comment}
区块链之上的加密数字货币无疑正在成为一种新的经济体~\cite{neweco},这一经济体将包含越来越多的行为及个体,整个系统也逐渐变得更加复杂,由最初的、单纯的数字货币系统,正在逐步成为智能合约、甚至应用的基础平台。

尽管区块链之上的加密数字货币是一种新型的经济体,我们认为,其仍然遵循必然的经济规律。也就是说,数字货币对应的经济体的变化仍然遵循经济学中的客观规律,
使用传统的、严谨的经济学方法对加密数字货币进行研究依然是有效的。

我们希望Nebulas能够成为一个高效的经济体,既,Nebulas能够被大家使用,用于各种场景下的交易、并支撑各种智能合约及应用。
因此,我们需要深入的理解一个经济体是如何变得高效的,以及如何定量的描述这些指标。

在传统的经济学中,存在很多指标用于描述经济系统,例如``流动性'',``流通性'',``货币流动性''等。
我们认为使用``货币流通速度''是一个便于使用的指标,Selden~\cite{selden}指出,货币流通速度是一个时期中货币的流量与该时期
中货币平均存量的比率。也就是说xxxx。
\subsection{宏观模型}

\end{comment}


\begin{comment}
\subsection{另一个版本}
PS:这里尝试从微观经济学来给出解释模型。

假设星云经济系统中,存在共计$N$个用户(或者称之为参与者,下文中两者表示同一概念)。用户$i$在任意时刻$t$需要发起数量为$x_{it}$的交易,后者$x_{it}$服从伯努利分布$B(1,p)$。对于用户$i$而言,其使用星云币进行交易的货币收益表示为$\frac{\widetilde{e}_{t+1}}{e}x_{it}$,其中$e$表示汇率,$\widetilde{e}_{t+1}$表示在t时刻星云经济系统仍然运行情况下的汇率,后者服从某个分布$G_{t+1}$。


在任意时刻$t$用户$i$选择策略$a_{it}\in \{T,\widehat{T}\}$,即使用星云币进行交易(表示为$T$),以及其他选择(例如通过其他交易渠道进行交易,表示为$\widehat{T}$)。在经济学中,每个参与者在参与博弈时采取不同行动策略会获取不同的收益,后者可以用收益函数(Payoff function)来表示。

综上,用户$i$交易行为的收益函数可以表示为:

\begin{align}
u(T;x_{it},e,\kappa_t,c_i)=\kappa_t \cdot \mathbb{E}_{it}v(\frac{\widetilde{e}_{t+1}}{e}x_{it})-c_i
\end{align}

其中$\kappa_t$表示在时刻$t$星云经济系统稳定的概率(该概率随时间增加而不断趋近于1,{\color{gray}相关证明后续需要的话会补充}),$c_i$表示用户交易产生的损失(例如额外的时间开销等),该损失存在上下限,因此对于任何用户$i$,有$c_i\in[c^-,c^+]$。
$\mathbb{E}_{it}$表示在给定环境下时刻$t$参与者$i$效用函数(utility function)的期望,需要指出的是,所有参与者均假设为风险厌恶倾向,因此用户效用函数用弱凸函数(weakly concave function)$v$表示,即效用随着货币收益增加而增加,但增加率递减。

对应用户$i$采取其他策略的收益函数表示为$u(\widehat{T};x_{it})$.

这里用$a_{it}$表示用户$i$在$t$时刻的决策,当$u(T;x_{it},e,\kappa_t,c_i)\geq u(\widehat{T};x_{it})$时,用户$i$选择采用星云币完成交易,即$a_{it}=T$,反之亦然。


因此市场需求可以表示为所有用户因为交易产生的星云币总量:
\begin{align}
\label{demandfunc}
Q_t(e;x_t)=\sum_ix_{it}1\{a_{it}=T\}=\sum_i x_{it}1\{u(T;x_{it},e,\kappa_t,c_i)\geq u(\widehat{T};x_{it})\}
\end{align}

市场供应量表示为常数$\bar{B}$,{\color{gray} 类似于上面所说$S/D$不变或者变化缓慢}

当达到市场均衡时,星云币的价格满足供给量与需求量相等。此时的经济系统满足如下特性:

1. 根据市场结清理论(market clearing),$e_t$是当需求小于供给时的最低汇率:

\begin{align}
\label{etfunc}
e_t=min\{e:Q_t(e;x_t)\leq\bar{B} e \}
\end{align}

2. 根据理性预期理论(rational expectations),在时刻$t$,参与者会充分利用所得信息来预期下一阶段的汇率分布$G_{t+1}$,并且该预期可被认为是合理正确的,这里用$\mu_{it}$表示用户$i$的理性预期。
\begin{align}
\label{me_func}
\mu_{it}=G_{t+1}
\end{align}



定理: 存在且唯一的汇率$e^{*}(x_t)$使得供需达到平衡.


证明:

首先在某个时刻$t>\hat{t}$,假设存在所有用户都选择使用星云币发起交易,我们可以找到满足此情况的一个充分必要条件:
\begin{align}
\mathbb{E}v(\frac{\widetilde{Z}}{N})-c^+>u(\widehat{T};1)
\end{align}

其中$\widetilde{Z}$是服从二项分布$Bi(N,p)$的随机变量。



汇率$e_t$则会趋近于:

\begin{align}
e_t=\sum_i x_{i,t} / \bar{B}
\end{align}

此时收益函数可以表示为:
\begin{align}
u(T;1,e,\kappa_t,c_i)=\kappa_t \cdot \mathbb{E}_{it}v(\frac{\widetilde{e}_{t+1}}{e})-c_i\geq \kappa_t \mathbb{E}_{it}v(\frac{\widetilde{e}_{t+1}}{N/\bar{B}})-c^+
\approx \kappa_t \mathbb{E}v(\frac{\widetilde{Z}}{N})-c^+>u(\widehat{T};1)
\end{align}

即此时对于所有参与者,使用星云币交易为最优选择。

根据上文,$G_{t+1}$表示$\widetilde{e}_{t+1}$的分布,在$\hat{t}$时刻,所有用户参与并使用星云币进行交易,因此$\widetilde{e}_{\hat{t}+1}$已知并且近似于$\widetilde{Z}/\bar{B}$的分布,即已知$G_{t+1}$,同理可以推出已知$G_t$。基于理性预期,市场需求可以根据公式\eqref{demandfunc}对每个用户累加求和得出。公式\eqref{demandfunc}为连续函数,且当$e$从0逐渐增大时,$Q_t(e;x_t)$从$\sum_ix_{it}$减少到趋近于0.这意味着,一定存在某个$e_t$满足公式\eqref{etfunc},即存在汇率$e_t$满足供需平衡。
$\square$


\end{comment}



为了操纵星云指数,
攻击者可以进行任意的操作,包括创建足够多的账户、进行账户之间的转账等,在诸多攻击方式中,唯一确定的事实是,
{\color{red} 用户需要将原本属于一个账户的资金拆分为多份,并转移到其他账户中},因此,为了抵抗操纵,
需要保证用户在拆分资金后,其星云指数会降低,
