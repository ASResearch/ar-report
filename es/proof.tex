% !TEX root = main.tex
\section{Pruebas}
\label{sec:appendix_proof}
\subsection{Prueba de propiedad \ref{prop:one}}
\begin{proof}
\label{proof:one}
Para todo $x_1>0$, $x_2>0$ tenemos:

\[
\begin{aligned}
f(x_1+x_2)&=\frac{x_1+x_2}{1+e^{a+b\cdot (x_1+x_2)}}\\
&=\frac{x_1}{1+e^{a+b\cdot (x_1+x_2)}}+\frac{x_2}{1+e^{a+b\cdot (x_1+x_2)}}\\
&=\frac{x_1}{1+e^{b\cdot x_2}\cdot e^{a+b\cdot {x_1}}}+\frac{x_2}{1+e^{b\cdot x_1}\cdot e^{a+b\cdot {x_2}}}\\
\end{aligned}
\]

En la ecuación \ref{eq:rank-param} tenemos $b<0$, de modo que $0<e^{b\cdot x_1}<1$, $0<e^{b\cdot x_2}<1$, por otra parte:

\[
\frac{x_1}{1+e^{b\cdot x_2}\cdot e^{a+b\cdot {x_1}}}>\frac{x_1}{1+ e^{a+b\cdot {x_1}}}=f(x_1)
\]

\[
\frac{x_2}{1+e^{b\cdot x_1}\cdot e^{a+b\cdot {x_2}}}>\frac{x_2}{1+ e^{a+b\cdot {x_2}}}=f(x_2)
\]

es, en realidad:
\[
f(x_1+x_2)>f(x_1)+f(x_2)
\]
\end{proof}

\subsection{Prueba de propiedad \ref{prop:two}}
\begin{proof}
Para todo $x_1>0$, $x_2>0$ tenemos:

\begin{equation}
\label{eq:fx_proof}
\begin{aligned}
f(x_1+x_2)-f(x_1)-f(x_2)&=\frac{x_1+x_2}{1+e^{a+b\cdot (x_1+x_2)}}-\frac{x_1}{1+e^{a+b\cdot x_1}}-\frac{x_2}{1+e^{a+b\cdot x_2}}\\
%&=\frac{x_1}{1+e^{a+b\cdot (x_1+x_2)}}+\frac{x_2}{1+e^{a+b\cdot (x_1+x_2)}}\\
&=(\frac{x_1}{1+e^{b\cdot x_2}\cdot e^{a+b\cdot {x_1}}}-\frac{x_1}{1+e^{a+b\cdot x_1}})\\
&\quad +(\frac{x_2}{1+e^{b\cdot x_1}\cdot e^{a+b\cdot {x_2}}}-\frac{x_2}{1+e^{a+b\cdot x_2}})\\
\end{aligned}
\end{equation}

Aquí la función $g(x_1,x_2)$ representa —dentro del segundo miembro de \ref{eq:fx_proof}— el primer término, y $h(x_1,x_2)$ el segundo término:

\begin{equation}
\label{eq:gx_func_proof}
g(x_1,x_2)=\frac{x_1}{1+e^{b\cdot x_2}\cdot e^{a+b\cdot {x_1}}}-\frac{x_1}{1+e^{a+b\cdot x_1}}
\end{equation}

\begin{equation}
\label{eq:hx_func_proof}
h(x_1,x_2)=\frac{x_2}{1+e^{b\cdot x_1}\cdot e^{a+b\cdot {x_2}}}-\frac{x_2}{1+e^{a+b\cdot x_2}}
\end{equation}

De modo que \eqref{eq:fx_proof} para $x_1$ y $x_2$, sus límites pueden ser representados como:
\[
\mathop {\lim }\limits_{\scriptstyle x_1 \to \infty  \hfill \atop  \scriptstyle x_2 \to \infty  \hfill}[f(x_1+x_2)-f(x_1)-f(x_2)]=\mathop {\lim }\limits_{\scriptstyle x_1 \to \infty  \hfill \atop  \scriptstyle x_2 \to \infty  \hfill}g(x_1,x_2)+\mathop {\lim }\limits_{\scriptstyle x_1 \to \infty  \hfill \atop  \scriptstyle x_2 \to \infty  \hfill}h(x_1,x_2)
\]

tenemos
\[
\begin{aligned}
g(x_1,x_2)&=\frac{x_1}{1+e^{b\cdot x_2}\cdot e^{a+b\cdot {x_1}}}-\frac{x_1}{1+e^{a+b\cdot x_1}}\\
&=\frac{x_1\cdot e^{a+b\cdot x_1}\cdot(1-e^{b\cdot x_2})}{(1+e^{b\cdot x_2}\cdot e^{a+b\cdot x_1})\cdot(1+e^{a+b\cdot x_1})}\\
&<\frac{x_1\cdot e^{a+b\cdot x_1}\cdot(1+e^{a+b\cdot x_1})}{(1+e^{b\cdot x_2}\cdot e^{a+b\cdot x_1})\cdot(1+e^{a+b\cdot x_1})}=\frac{x_1\cdot e^{a+b\cdot x_1}}{1+e^{b\cdot x_2}\cdot e^{a+b\cdot x_1}}\\
&<\frac{x_1\cdot e^{a+b\cdot x_1}}{1+e^{a+b\cdot x_1}}=\frac{x_1}{1+\frac{1}{e^{a+b\cdot x_1}}}
\end{aligned}
\]

Se calcula el límite para $\frac{x}{1+\frac{1}{e^{a+b\cdot x}}}$, de acuerdo a la regla de L'Hôpital:
\[
\begin{aligned}
\lim_{x \to \infty}\frac{x}{1+\frac{1}{e^{a+b\cdot x}}}&=\lim_{x \to \infty}\frac{1}{(e^{-a-b\cdot x})'}\\
&=\lim_{x \to \infty}\frac{1}{-b\cdot e^{-a-b\cdot x}}
\end{aligned}
\]

En la ecuación \ref{eq:rank-param} tenemos $b<0$, por lo que $\lim_{x \to \infty}-b\cdot e^{-a-b\cdot x}=\infty$; por otra parte,

\[
\begin{aligned}
\lim_{x \to \infty}\frac{x}{1+\frac{1}{e^{a+b\cdot x}}}=0
\end{aligned}
\]

De acuerdo a \ref{proof:one}, tenemos $g(x_1,x_2)>0$, por lo que de acuerdo al teorema del emparedado:

\[
\mathop {\lim }\limits_{\scriptstyle x_1 \to \infty  \hfill \atop  \scriptstyle x_2 \to \infty  \hfill}g(x_1,x_2)=0
\]

Similarmente, podemos obtener:
\[
\mathop {\lim }\limits_{\scriptstyle x_1 \to \infty  \hfill \atop  \scriptstyle x_2 \to \infty  \hfill}h(x_1,x_2)=0
\]

Por lo que:
\[
\mathop {\lim }\limits_{\scriptstyle x_1 \to \infty  \hfill \atop  \scriptstyle x_2 \to \infty  \hfill}[f(x_1+x_2)-f(x_1)-f(x_2)]=0
\]


\end{proof}