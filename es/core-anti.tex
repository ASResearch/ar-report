% !TEX root = main.tex

\section{Manipulation-resistance of Core Nebulas Rank}

This chapter is the analysis on how Core Nebulas Rank resists manipulation, i.e. the fairness of Nebulas Rank.

\emph{Manipulation} refers to dishonest actions of an attacker for personal gain. Actions that may be undertaken by attackers include: launching asset transfers, this involves making use of assets and accounts controlled by them and their other dishonest individuals. Among the transfers, the amount of asset doesn't exceed the asset owned by the attacker; the source of transfer is either the accounts owned by the attacker and its cooperators, or some institutes' accounts who serve as exchanges. Usually, the benefit obtainable is determined by the accounts whose private keys are known by the attacker. A simple case is that the attacker's benefit is the sum of all of these accounts' ranking scores. Of course, it could be noticed that the private keys of institutes' accounts mentioned before are not controlled by the attacker.

The analysis of this section is based on the actions undertaken and the attackers' benefit defined above. First, we discuss the upper-bound for a single account's ranking score enhancement. Then we analyze the upper-bound for multiple accounts. Last, collusion is included and we discuss the situation of more than one attacker.

\subsection{Ranking Score Enhancement for One Account \label{sec:cheat-single}}

According to \refeq{eq:rank-param} in order to raise the ranking score for one account, the ranking score of the account is positively correlated with both the amount of assets and the in-and-out degree. The amount of assets in the account, i.e. $\beta$, has an upper bound, i.e. it is no more than the absolute total of the assets owned by the attacker, denoted by $\beta_0$. And in-and-out degree $\gamma$ represents the volume of transfers, which means the attacker needs to increase the transfers amount of one controlled account as much as possible.

The increasing of transfer amount includes two parts: increasing in-degree and increasing out-degree. Increasing in-and-out degree needs two participating accounts, one of which is the target account whose goal is to raise their ranking score, the other account could either be a controlled account or an uncontrolled account. If it is an uncontrolled account, increasing degree means transacting with other people, this situation is discussed in \refsec{sec:coalition}. The other case is that the attacker sends assets to strangers unconditionally, which is too costly that it won't be discussed in this section. Therefore typically, it could be defined that, the actions of attackers mainly focus on increasing the transfers among the accounts controlled by themselves. Since the assets controlled by attackers are limited and the time period for ranking is also limited, it holds that the degree of an account has an upper-bound which is decided by the amount of assets held by the attacker.

As analyzed above, we consider the scenario of transacting with accounts of the
same owner. Based on the computation method \refeq{eq:rank-param} as defined in \refsec{sec:function}, the attacker's benefit will be reduced if it split the asset transfers into multiple ones. Thus the attacker will attempt to make its transaction amount to be as high as possible, i.e. it tries to transfer all assets it owns into the account and then transfer it out all. Due to the cycle-removal algorithm, the attacker's asset cannot be transferred in again during this period. And the in-and-out degree is $\gamma = 2 \beta_0$. The ranking score is
\begin{align}
\mathcal{C} =  \frac{2 \beta_0 ^2}{ (1+e^{a + b \cdot \beta_0}) (1+e^{c + 2 d
  \cdot \beta_0})}.
\end{align}

Additionally, we consider a more advanced manipulation technique. Consider the case that an attacker manages to acquire the asset again somewhere else by transacting off-line. Then it could transfer the asset into the account again and the upper-bound of in-and-out degree is the asset amount times the number of off-line transactions. Since the ranking time period is limited, the upper-bound of the number of off-line transactions is a constant integer, i.e. $\gamma$ is bounded by $2T \cdot \beta_0$, where $T$ is a constant integer indicating the length of ranking time period. Therefore the upper-bound score is
\begin{align}
  \mathcal{C} =  \frac{2T \cdot \beta_0 ^2}{ (1+e^{a + b \cdot \beta_0})
  (1+e^{c + c \cdot d \cdot \beta_0})}.
\end{align}

\subsection{Ranking Score Enhancement for Multiple Accounts (Sybil Attack)}
Sybil Attack refers to a situation whereby the attacker obtains falsely high ranking score by creating a large number of pseudo-identities to tamper the reputation system of P2P network ~\cite{quercia2010sybil}.

An entity on a peer-to-peer network is a piece of software which has been granted access to local resources. An entity advertises itself on the peer-to-peer network by presenting an identity. More than one identity can correspond to a single entity. In other words, the mapping of identities to entities may be multiples to one. Entities in peer-to-peer networks utilise multiple identities for purposes of redundancy, resource sharing, reliability and integrity. In peer-to-peer networks, the identity is used as an abstraction so that a remote entity can be aware of identities without necessarily knowing the correspondence of identities to local entities. By default, each distinct identity is usually assumed to correspond to a distinct local entity. In reality, many identities may correspond to the same local entity. An adversary may present multiple identities to a peer-to-peer network in order to appear and function as multiple distinct nodes. The adversary may thus be able to acquire a disproportionate level of control over the network, such as by affecting voting outcomes ~\cite{wiki:sybil}.

Here we assume the attacker's payoff is the sum of all accounts controlled by the attacker. Considering the strategy to enhance ranking score for one account, which is analyzed at last subsection, the attacker could apply the same strategy to multiple accounts: starting from any one account, the attacker transfer part of its asset into the next account, finally forming a linked asset flow. In this case, since Core Nebulas Rank requires that no more than valid amount of asset stays in the account for no more than half of the period, by no means the attacker could make $\beta$ for more than one account to be the total amount of assets owned by it. Thus the attacker should adopt another strategy where its assets are evenly distributed into all its accounts. Suppose the length of link is $N$, i.e. there are $N$ controlled accounts, and for every account, $\beta = \frac{\beta_0}{N}$. The in-and-out degree analysis is same with \refsec{sec:cheat-single}, the upper-bound of $\gamma$ is $K \cdot \beta$, where $K=2\cdot N$ is a constant integer. Therefore the upper-bound of the sum of all accounts owned by the attacker is:

\begin{align}
\mathcal{C} = N \cdot \frac{K \frac{\beta_0 ^2}{N}}{ (1+e^{a + b \cdot \frac{\beta_0}{N} }) (1+e^{c + K \cdot d \cdot \beta_0})} = \frac{K \beta_0 ^2 }{ (1+e^{a + b \cdot \frac{\beta_0}{N} }) (1+e^{c + K \cdot d \cdot \beta_0})}
\end{align}


\subsection{Coalition Manipulation \label{sec:coalition}}
The result of coalition manipulation is no different than the case where one attacker owns the original total assets of two attackers. In this situation we can analyze the case of coalition manipulation by analyzing the consequence of a single attacker's assets increasing.



