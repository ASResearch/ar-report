
\section{区块链经济模型}
\begin{frame}
	\frametitle{加密数字货币表示}
  加密数字货币系统,表示为$(\mathcal{L}, \mathcal{U})$,其中
  \begin{align}
  \mathcal{L} = (\mathcal{A}, \mathcal{D}, \mathcal{T})
  \end{align}
  \begin{align}
  \mathcal{D} = \{a \rightarrow d, a{\in}\mathcal{A}, d{\in}R^*\}
  \end{align}
  \begin{align}
  \mathcal{T} = \{(s, t, w, \tau)\}
  \end{align}
一个账户与用户只能被唯一的用户控制,一个用户能够控制多个账户
\begin{align}
u \dom a. \quad u\in \mathcal{U}, a\in \mathcal{A}
\end{align}
\begin{align}
A(u) = \{\forall a\in \mathcal{A} : u \dom a\}
\end{align}
\begin{align}
\forall u_1, u_2 \in \mathcal{U} : A(u_1) \cap A(u_2) = \phi
\end{align}
\end{frame}

\begin{frame}
\frametitle{加密数字货币模型}
数字货币应该具备货币属性,即交换媒介、计价单位和价值储藏手段。
经典货币数量论:
\begin{align}
M\times V=P\times Y
\label{eq:currency}
\end{align}
而在加密数字货币中,可以写为:
\begin{align}
(M + \Delta{m}) \times \frac{\sum_{(s, t, w, \tau)\in \mathcal{T}}{w}}{M} = P \times Y
\label{eq:cur_ext}
\end{align}
更进一步的
\begin{align}
Y=\sum_{a\in \mathcal{A}} \mathcal{C}(a)
\end{align}
\noindent 其中$\mathcal{C}(a)$表示账户$a$对整个经济产量的贡献,即星云指数。
\end{frame}

\begin{frame}
\frametitle{星云指数}
星云指数衡量了一个账户对经济系统的贡献。数字货币的发展依赖于社区的发展,因
此,我们认为量化社区中每个账户对于经济总量的贡献,为正确的激励提供了必要的基
础。
\begin{itemize}[<+->]
\item 核心星云指数
\item 扩展星云指数
\end{itemize}
\end{frame}
